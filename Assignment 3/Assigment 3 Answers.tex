\documentclass[11pt, a4paper, answers]{exam}

\usepackage{amsmath}
\usepackage[utf8]{inputenc}
\usepackage{tikz}



\begin{document}
\begin{center}
    {\large \textbf{Cosmology PHYS 24010 Assessed sheet 1.}}\\
    Samuel. A. Hopkins - 1942842 \\
    April 2021

\end{center}
\begin{questions}
    \unframedsolutions
    \question Determine the electron to baryon ratio in the Universe, noting any assumptions.

    \begin{solution}
        The two main assumptions for this solution are as follows;
        \begin{itemize}
            \item The Universe is charge neutral $\to$ for every proton, there is one electron
            \item The Universe is mainly comprised of $^4$He and Hydrogen, and for every 1 nuclei
                  (in this case, an atom) of helium there are 12.7 hydrogen atoms(on average).
        \end{itemize}
        We know $^4$He has 2 neutrons, 2 protons and 2 electrons for it to be charge neutral and
        hydrogen must have 1 proton and 1 electron. We can then sum the total number of electrons
        and baryons (neutrons and protons) from this ratio as;
        \begin{equation*}
            \frac{12.7+2}{12.7+4} = 0.88
        \end{equation*}
        where the numerator is comprised of the 12.7 electrons from hydrogen and the 2 electrons
        from helium. The denominator is comprised of the 4 baryons from helium and 12.7 baryons from
        protons in hydrogen.
        Therefore the electron to baryon ratio is $0.88$, i.e for every electron, there are $0.88$
        baryons on average.
    \end{solution}

    \begin{center}
        \rule{8cm}{0.4pt}
    \end{center}
    \newpage

    \question Determine the neutron to proton ratio 2 minutes after the Big Bang, describing and
    justifying each step in your calculation.

    \begin{solution}
        We first need to calculate the neutron-proton ratio when protons and neutrons stopped
        converting into each other, i.e when the temperature cools enough to $ k_BT = 0.8$MeV. We
        calculate this using the following equation;
        \begin{equation*}
            \frac{N_n}{N_p} = \left( \frac{m_n}{m_p}\right)^{3/2}\text{exp}\left[
                -\frac{(m_n-m_p)c^2}{k_{\text{B}}T}
                \right]
        \end{equation*}
        the ratio of the masses in the equation is close to 1, so we can make good approximation
        just using the exponential term like so;
        \begin{equation*}
            \frac{N_n}{N_p} \approx\text{exp}\left[
                -\frac{(m_n-m_p)c^2}{k_{\text{B}}T}
                \right]
        \end{equation*}
        the difference in masses is $1.3MeV$ and we use the temperature at which nucleons stop
        converting into eachother ($0.8MeV$) to obtain;
        \begin{equation*}
            \frac{N_n}{N_p} \approx \text{exp}\left[
                \frac{1.3 \text{MeV}}{0.8 \text{MeV}}
                \right] \approx 0.196911 \approx \frac{1}{5.1}
        \end{equation*}
        After this temperature of $ k_BT = 0.8$MeV, we can calculate the decrease in the number of
        neutrons due to free-neutron decay through;
        \begin{equation*}
            \text{exp}\left[-\text{ln}(2)\times (t/t_{\text{half}})\right]
        \end{equation*}
        we know the half life of free-neutrons as 614s, and the time we want to calculate the ratio
        at is $t = 120$s, so the ratio becomes;
        \begin{equation*}
            \frac{N_n}{N_p} \approx \frac{1}{5.1} \times \text{exp}\left[
                -\frac{\text{ln}2 \times 120}{614}
                \right] \approx 0.175 \approx \frac{1}{5.7}
        \end{equation*}

    \end{solution}


    \begin{center}
        \rule{8cm}{0.4pt}
    \end{center}
    \newpage


    \question Describe the processes that give rise to the Cosmic Microwave Background, indicating why
    its discovery ruled out a steady-state cosmology ($<$ 200 words).

    \begin{solution}
        Initially, the Universe consisted of a hot, dense plasma of photons, quarks and leptons
        which presented itself as an opaque ‘fog’, as the photons at the time could not travel
        far without encountering matter. It eventually expanded and cooled enough to allow
        quarks to bind together to form hadrons. When the Universe had cooled further, at the
        recombination epoch, it allowed protons and electrons to bind together to form neutral
        hydrogen atoms as the photon energy$<$nucleon binding energy. By decoupling, the mean
        free path of photon was long enough to allow the photons to travel long distances, as
        the hydrogen that was formed, was transparent and any photons that were not captured
        by hydrogen atoms were able to travel large distances, to the point where we can detect
        them today as the CMB.
        \newline
        The Big-Bang model predicted that the Universe should be filled with a relic radiation
        from the big bang, that could be detected today as the CMB. The steady-state model provided
        no feasible explanation of the existence of the CMB and attributed it to light from ancient
        stars that scattered by dust. The isotropy of the CMB disproves this, however.

    \end{solution}



    \begin{center}
        \rule{8cm}{0.4pt}
    \end{center}
    \newpage


    \question Use Wien's law to explain how the CMB's temperature changes with scale factor of the
    Universe, $a$, and describe how it's energy density changes $a$. Explain why it is estimated
    that the temperature was about 3300 K at decoupling. Determine the CMB's energy density at the
    time of decoupling demonstrate that $a = 0.00083$ at that time.
    \begin{solution}

        \underline{How Temperature and Energy density scales with scale factor:}

        Starting from Wien's Law;
        \begin{equation*}
            \lambda_{\text{peak}} = \frac{2.898\times10^{-3}}{T}
        \end{equation*}
        where $T$ is the temperature of the CMB. We know that the wavelength of blackbodies
        stretches with scale factor like so: $\lambda_{\text{peak}}\propto a(t)$. So we can
        immediately see that from Wien's law;
        \begin{equation*}
            T \propto \frac{1}{a(t)}
        \end{equation*}
        We also know that from the planck function, the energy density of radiation is given by;
        \begin{equation*}
            \epsilon_{\gamma} = \rho_{\gamma}c^2 = \alpha T^4
        \end{equation*}
        from this, we can see the relation between temperature and energy density and thus scale
        factor;
        \begin{equation*}
            \epsilon_{\gamma} \propto \frac{1}{a(t)^4}
        \end{equation*}
        You can also arrive at the same conclusion by looking at the relation between the radiation
        density from the fluid equation, as we know $\rho_{\gamma} = \rho_0/a^4$, and energy density
        is given by;
        \begin{align*}
            \epsilon_{\gamma} =                  & \rho_{\gamma}c^2              \\
            =                                    & \frac{\rho_{\gamma_{0}}}{a^4} \\
            \implies  \epsilon_{\gamma}  \propto & \frac{1}{a^4}
        \end{align*}
        \underline{Estimation of Temperature at Decoupling:}

        At decoupling, it is estimated that the temperature was around 3300K. This however is
        different to the value you would get if you calculate the temperature using the mean energy
        of the CMB photons like so;
        \begin{equation*}
            T \approx \frac{13.6 eV}{3k_{\text{B}}} = 50000 K
        \end{equation*}
        This is a very big difference as this does not consider the fact that only a small
        proportion of the photon would have to have photon energies $>$13.6 eV to keep hydrogen
        ionised. This is due to the fact that the photons form a blackbody spectra and there would
        be sufficient photons in the high-energy exponential tail to fulfill this
        condition, even if the mean energy of the photons was much lower. This leads to a lower
        energy estimation of 3300K.

        \underline{Determining the Energy density and scale factor at decoupling}

        At decoupling, $T = $3300K. Can calculate the energy density using the following relation;
        \begin{equation*}
            \epsilon_{\gamma} = \alpha T^4
        \end{equation*}
        where $\alpha$ is;
        \begin{equation*}
            \alpha = \frac{\pi^2k^4_{\text{B}}}{15\hbar^3c^3} \approx 7.57\times 10^{-16} \: \text{J}
            \: \text{m}^{-3} \: \text{K}^{-4}
        \end{equation*}
        so the energy density is therefore;
        \begin{equation*}
            \epsilon_{\gamma} = 7.57\times 10^{-16} \times (3300)^4  = 0.0898 \: \text{J}
            \: \text{m}^{-3}
        \end{equation*}
        To demonstrate that the scale factor at decoupling was $a = 0.00083$, we use the
        relation from before;
        \begin{equation*}
            \epsilon_{\gamma} = \frac{\rho_{\gamma_{0}}c^2}{a^4}
        \end{equation*}
        where $\rho_{\gamma_{0}}$ is the present day energy density which can be calculated from the
        current CMB temperature (2.725K);
        \begin{align*}
            \epsilon_{\gamma} = \rho_{\gamma_{0}}c^2 = \alpha T^4 \\
            \implies \rho_{\gamma_{0}}c^2 = \alpha T^4            \\
        \end{align*}
        rearranging the above we obtain;
        \begin{equation*}
            \rho_{\gamma_{0}} = \frac{\alpha T^4}{c^2} =
            \frac{7.57\times 10^{-16} \times (2.725)^4}{c^2} = 4.64 \times 10^{-31} \text{kg} \:
            \text{m}^{-3}
        \end{equation*}
        Now we simply rearrange the equation from before like so;
        \begin{align*}
            \epsilon_{\gamma} = & \frac{\rho_{\gamma_{0}}c^2}{a^4} \\
            a =                 & \left(
            \frac{\rho_{\gamma_{0}} c^2}{\epsilon_{\gamma}}
            \right)^{\frac{1}{4}} = \left(
            \frac{4.64\times 10^{-31} c^2}{0.0898}
            \right) = 8.26 \times 10^{-4}                          \\
            \to a =             & 0.00083
        \end{align*}
        for $a$ at decoupling.
    \end{solution}

    \begin{center}
        \rule{8cm}{0.4pt}
    \end{center}
    \newpage


    \question If the current value of the matter density parameter is $\Omega_M = 0.3$ and the Hubble constant
    is 70 km s$^{-1}$ Mpc$^{-1}$, find the value of the scale factor at equipartition (when the radiation and
    matter energy densities were equal). Compare with the value of $a$ at decoupling, and deduce
    which happened first.

    \begin{solution}
        At equipartition, $\rho_m = \rho_{\gamma}$, we also know $\rho_m = \frac{\rho_{m_0}}{a^3}$
        and $\rho_{\gamma} = \frac{\rho_{\gamma_0}}{a^4}$, so we can write the following relation;
        \begin{equation*}
            \frac{\rho_{m_0}}{a^3} = \frac{\rho_{\gamma_0}}{a^4}
        \end{equation*}
        leading to;
        \begin{equation*}
            \frac{\Omega_{m_0}}{\Omega_{\gamma_0}} = \frac{1}{a}
        \end{equation*}
        of which, we know $\Omega_{m_0}$, so we need to determine $\Omega_{\gamma_0}$. This can be
        calculated from the total energy density equation;
        \begin{equation*}
            \epsilon_{\gamma_0} = \rho_{\gamma_{0}}c^2 = \alpha T^4
        \end{equation*}
        We use this to calculate the present day energy density;
        \begin{equation*}
            \epsilon_{\gamma_0} = \alpha \times(2.725)^4 = 4.174\times 10^{-14} \:\text{J} \:\text{m}^{-3}
        \end{equation*}
        from this we can calculate the present day density parameter for radiation by using
        $\rho_{\gamma_{0}} = \Omega_{\gamma_0}\rho_c$ and inputting into the equation for the total
        energy density;
        \begin{equation*}
            \epsilon_{\gamma_0} =  \Omega_{\gamma_0}\rho_c c^2
        \end{equation*}
        rearranging this, where $h=0.7$, we obtain;
        \begin{equation*}
            \Omega_{\gamma_0} = \frac{\epsilon_{\gamma_0}}{\rho_c c^2} =
            \frac{4.174\times 10^{-14}}{1.88h^2\times10^{-26}\times(3\times 10^8)^2} =
            2.467 \times10^{-5}h^{-2}
        \end{equation*}
        from before;
        \begin{align*}
            \frac{\Omega_{m_0}}{\Omega_{\gamma_0}}          & = \frac{1}{a} \\
            \implies \frac{\Omega_{\gamma_0}}{\Omega_{m_0}} & = a
        \end{align*}
        and finally;
        \begin{equation*}
            \frac{2.467 \times10^{-5}h^{-2}}{0.3} = 1.678\times10^{-4} = 0.0001678
        \end{equation*}
        The scale factor at decoupling $a = 0.00083$ corresponds to a redshift of $z \approx
            1203.8$, compared to equipartition where its scale factor $a = 0.0001678$ corresponds to a
        redshift of $z \approx 5958$, indicating that equipartition occured first.
    \end{solution}

    \begin{center}
        \rule{8cm}{0.4pt}
    \end{center}
    \newpage


    \question For this question adopt a scale factor of $a = 1/1200$ for decoupling, the time at which the
    Cosmic Microwave Background radiation originated. Adopt ages for the Universe of $4\times 10^5$
    years at decoupling and $1.4\times10^{10}$ years for the present day. Consider the light-travel distance
    from the Big Bang to the time of decoupling. What is that distance today, $r$, taking into
    account the change in scale factor of the Universe? Combine $r$ with the distance light can
    have travelled since decoupling and use simple trigonometry to find an angular size in units of
    degrees that light could have travelled in the plane of the sky.

    \begin{solution}
        We know the speed of light in convienient units as, $3\times10^{-7}\:\text{Mpc}\:
            \text{yr}^{-1}$.
        The time from the big bang to decoupling is given, so we can calculate the light-travel
        distance at decoupling as $(3\times10^{-7}) \times (4\times 10^5) = 0.12$ Mpc. Now
        taking into account the scale factor of the universe, where $d$ is the light-travel
        distance at decoupling and $r$ is the stretched distance;
        \begin{equation*}
            r = \frac{d}{a} = \frac{0.12}{1/1200} = 144 \:\text{Mpc}
        \end{equation*}

        \tikzset{every picture/.style={line width=0.75pt}} %set default line width to 0.75pt        

        \begin{tikzpicture}[x=0.75pt,y=0.75pt,yscale=-1,xscale=1]
            \path (75,300); %set diagram left start at 0, and has height of 300

            %Shape: Triangle [id:dp9345497653845598] 
            \draw   (250.44,332.55) -- (433.35,101.78) -- (496.33,170.53) -- cycle ;
            %Straight Lines [id:da9909509960860299] 
            \draw    (250.44,332.55) -- (463.92,138.35) ;
            \draw [shift={(465.4,137)}, rotate = 497.71] [color={rgb, 255:red, 0; green, 0; blue, 0 }  ][line width=0.75]    (10.93,-3.29) .. controls (6.95,-1.4) and (3.31,-0.3) .. (0,0) .. controls (3.31,0.3) and (6.95,1.4) .. (10.93,3.29)   ;
            %Straight Lines [id:da16201400521356324] 
            \draw    (470.07,132.33) -- (438.7,97.92) ;
            \draw [shift={(437.35,96.44)}, rotate = 407.65] [color={rgb, 255:red, 0; green, 0; blue, 0 }  ][line width=0.75]    (10.93,-3.29) .. controls (6.95,-1.4) and (3.31,-0.3) .. (0,0) .. controls (3.31,0.3) and (6.95,1.4) .. (10.93,3.29)   ;
            %Straight Lines [id:da9495619209810333] 
            \draw    (470.07,132.33) -- (499.64,164.4) ;
            \draw [shift={(501,165.87)}, rotate = 227.31] [color={rgb, 255:red, 0; green, 0; blue, 0 }  ][line width=0.75]    (10.93,-3.29) .. controls (6.95,-1.4) and (3.31,-0.3) .. (0,0) .. controls (3.31,0.3) and (6.95,1.4) .. (10.93,3.29)   ;
            %Shape: Arc [id:dp7656237137008544] 
            \draw  [draw opacity=0] (320.93,243.38) .. controls (323.05,242.67) and (325.42,242.36) .. (327.89,242.56) .. controls (334,243.04) and (339.01,246.48) .. (340.9,250.96) -- (326.94,254.52) -- cycle ; \draw   (320.93,243.38) .. controls (323.05,242.67) and (325.42,242.36) .. (327.89,242.56) .. controls (334,243.04) and (339.01,246.48) .. (340.9,250.96) ;

            % Text Node
            \draw (408.67,186.4) node [anchor=north west][inner sep=0.75pt]    {$D$};
            % Text Node
            \draw (476.67,110.73) node [anchor=north west][inner sep=0.75pt]    {$r$};
            % Text Node
            \draw (300.33,225.07) node [anchor=north west][inner sep=0.75pt]    {$\frac{\theta }{2}$};


        \end{tikzpicture}

        We can then use simple trigonometry in order to determine the angular size of the light
        travel distance, as shown in the figure above. We first need to determine $D$, the distance light has travelled since
        decoupling: $(3\times10^{-7}) \times (1.4\times 10^{10} = 4200)$Mpc. $\Theta$ will then be
        given by;

        \begin{equation*}
            \theta = \text{arctan} \left(
            \frac{144}{4200}
            \right) = 1.96^\circ \approx 2^\circ
        \end{equation*}
        \[

        \]
    \end{solution}


    \begin{center}
        \rule{8cm}{0.4pt}
    \end{center}
    \newpage


    \question Describe ($<$ 200 words) two main problems with Big Bang cosmology that are solved by
    cosmological inflation. Describe ($<$ 200 words) without equations the concept of inflation and
    how it solves these two problems

    \begin{solution}
        The two main problems with Big Bang cosmology are the flatness problem, and the horizon
        problem.
        \newline

        \underline{The Horizon Problem}

        At decoupling, we expect different patches of blackbody radiation to be at different
        temperatures. We would expect to see this difference as the universe expands, however we
        know the CMB is isotropic, so we infer that thermalization occurred before decoupling.
        This again, is a problem as the light travel distance at this point is too short for regions
        to interact thermally, to the point where regions more than degree or two cannot interact
        to exchange photon. This is known as the horizon problem, as how could two regions of space,
        causally disconnected, be the same temperature?
        \newline

        \underline{The Flatness Problem}

        As our universe is incredibly close to being flat, as told by the total density parameter.
        We can infer that the universe had to have been even closer to flatness at earlier times,
        as deviations from unity from the total density parameter would rapidly increase with time,
        leading to much larger deviations than observed today. This leads us to question how the
        Universe sets the total density parameter so close to unity at such early times.

        \underline{Inflation and it's solutions}

        Inflation is described as a period in the evolution of the Universe where the scale factor
        was accelerating, leading to an exponential expansion of space. This period was relatively
        short in the evolution of the universe as it had to end before nucleosynthesis, as those
        processes would have been affected by the rapid expansion, due to the dominance of the
        cosmological constant.  This solves both the flatness and horizon problems with the big
        bang.

        The horizon problem is solved, as inflation greatly increases the size of a region of the
        Universe, to the point where small regions initially able to achieve thermalization before
        inflation are then inflated to the point where they could be larger than the size of the
        observable Universe.

        To solve the flatness problem, inflation forces the deviation from
        unity in the total density parameter to zero. This means that even if the deviation is set
        at any arbitrary value, the period of inflation essentially resets it back to zero and
        leaves it an extremely small value. This allows the expansion of the Universe to cause this
        value to grow to a value which allows the Universe to remain flat, as seen from the
        observable Universe.

    \end{solution}

    \begin{center}
        \rule{8cm}{0.4pt}
    \end{center}
    \newpage


    \question If the best current measurements give $\Omega_{tot_0}=1\pm 0.1$ and assuming that
    inflation started at $t = 10^{-36}$s after the Big Bang, determine the earliest time that
    inflation can have stopped and still be in agreement with that measurement if $\Omega_{tot}$
    prior to inflation to 0.001. Assume that the Universe was radiation dominated for the entire
    time subsequent to the inflation epoch (not true, but OK for the purposes of the question) and
    is $4\times10^{17}$s old.

    \begin{solution}
        We know from the current conditions for inflation that;
        \begin{equation*}
            \frac{|\Omega_{\text{tot}}(t)-1|_{\text{after}}}{|\Omega_{\text{tot}}(t)-1|_{\text{before}}} = \left(
            \frac{a_1}{a_2}
            \right)^2 \approx 10^{-86}
        \end{equation*}
        and we are given, $\Omega_{\text{tot}} = 0.001 $ which then provides us with the condition;
        \begin{equation*}
            |\Omega_{\text{tot}}(t)-1|_{\text{after}} \le 10^{-86}
        \end{equation*}
        as the denominator became $|0.001 - 1|_{\text{before}} = 0.999\approx 1$
        We know that for radiation dominated universe, $|\Omega_{\text{tot}}(t)-1|_{\text{after}}
            \propto t$ and thus;
        \begin{equation*}
            |\Omega_{\text{tot}}(t)-1|_{\text{after}} \propto \frac{1}{a^2}
        \end{equation*}
        inferring that the scale factor must change by $10^{43}$. From this we can use the following
        relation to find the time at which inflation could have stopped for this condition;
        \begin{align*}
            \frac{a_2}{a_1} = & \text{exp}\left[
                \frac{t_{\text{end}}-t_1}{\tau}
            \right]                              \\
            10^{43} =         & \text{exp}\left[
            \frac{t_{\text{end}}-10^{-36}}{10^{-36}}
            \right]
        \end{align*}
        taking the natural log of both sides, we obtain;
        \begin{align*}
            \text{ln}(10^{43}) = \frac{t_{\text{end}}-10^{-36}}{10^{-36}} \\
            \to t_{\text{end}} = \text{ln}(10^{43}) \times 10^{-36} + 10^{-36}
        \end{align*}
        which gives us a value of $t_{\text{end}} = 1\times10^{-34}$s
    \end{solution}

    \begin{center}
        \rule{8cm}{0.4pt}
    \end{center}
    \newpage


    \question Explain what is meant by the strong energy condition and how it constrains the dynamical
    evolution of the Universe. Demonstrate using the Friedmann and acceleration equations that a
    universe with a cosmological constant, but otherwise empty does not require a Big Bang. What
    can be said about the age of that universe? Determine whether or not such a universe obeys
    the strong energy condition.

    \begin{solution}
        The strong energy condition is given as;
        \begin{equation*}
            \rho c^2 + 3p \ge 0
        \end{equation*}
        should the Universe obey this condition, it will be constrained to always be decelerating.
        The Friedmann equation is given as;
        \begin{equation*}
            \left(\frac{\dot{a}}{a}\right)^2 = \frac{8\pi G}{3}\rho - \frac{kc^2}{a^2} + \frac{\Lambda}{3}
        \end{equation*}
        The accleration equation;
        \begin{equation*}
            \frac{\ddot{a}}{a} = -\frac{4\pi G }{3}\left(\rho+ \frac{3p}{c^2}\right) + \frac{\Lambda}{3}
        \end{equation*}
        For an empty universe, the friedmann and acceleration equations evaluate to;
        \begin{align*}
            \left(\frac{\dot{a}}{a}\right)^2 = \frac{\Lambda}{3} \\
            \frac{\ddot{a}}{a} = \frac{\Lambda}{3}
        \end{align*}
        It can be seen from the above that this violates the strong energy condition as the
        accleration equation yields a positive acceleration, for a non-zero cosmological constant.
        Solving the Friedmann equation by separation of varibles;
        \begin{align*}
            \int^a_0\frac{da}{a} = \int^t_0\sqrt{\frac{\lambda}{3}}dt
        \end{align*}
        gives the result;
        \begin{align*}
            \text{ln}(a) = \sqrt{\frac{\Lambda}{3}}t \\
            \to a = \text{exp}\left[\sqrt{\frac{\Lambda}{3}}t\right]
        \end{align*}
        This indicates that even at start of the universe, ($t = 0$), the scale factor is non-zero
        and thus there was never a singularity(big-bang) for a universe such as this.
        The age of this universe would be simply given by the Hubble time, $H_0^{-1}$.
    \end{solution}
    \begin{center}
        \rule{8cm}{0.4pt}
    \end{center}
    \newpage


    \begin{align*}
        2d_n = (n+\frac{1}{2})\lambda_0; & \qquad{\text{bright fringes}} \\
        2d_n = n\lambda_0;               & \qquad{\text{dark fringes}}
    \end{align*}


\end{questions}
\end{document}