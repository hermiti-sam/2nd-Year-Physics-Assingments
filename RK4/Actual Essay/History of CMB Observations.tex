\documentclass[twocolumn, prl, nobalancelastpage, aps, citeautoscript, longbibliography, 10pt]{revtex4-1}
\usepackage{geometry, graphics, bm, tikz, amsmath}
\renewcommand{\baselinestretch}{0.95}

\begin{document}

\title{The History of the Cosmic Microwave Background Observations}
\author{S. A. Hopkins}
\noaffiliation
\date{\today}

\maketitle
\section{Introduction}
The Cosmic Microwave Background (CMB) radiation is relic radiation from the early stages of the Universe, first discovered by two 
American radio astronomers, Penzias and Wilson in 1965. The CMB is electromagnetic radiation, that is almost uniform in all 
directions(isotropic) and it's value has been measured at $T = 2.72548\pm0.00057$K\cite{Fixsen}. It was first postulated by Gamow, Aplher and 
Herman in the 1940's, stating that in order for nucleosynthesis of 
light elements (Helium, Hyrdogen and lithium) to have happened, the universe would have been incredibly hot, which would have left 
radiation detectable today. The leftover radiation was found at the Holmdel Horn Antenna by Penzias and Wilson and served as a game-changing
development in modern cosmology, providing cosmologists a tool for understanding the dyanmics and geometry of our universe. 
Since then, many theoretical and observational studies, notably the WMAP and Planck missions have established a standard cosmological model 
that uses the CMB, to describe our universe as a flat universe, that is mainly comprised of non-baryonic, dark matter and energy that serves
to accelerate the expansion of the universe. 




\section{Oberservations}
\subsection{Excess Antenna Temperature and it's interpretation}
In 1965, \textit{The Astrophysical Journal} published two papers detailing the discovery and correct interpretation of the CMB.
The first paper was authored by Penzias and Wilson, where they defined the measurement of a constant residual noise
in their equipment at the Holmdel Horn Antenna that was evenly spread across the sky. The 'noise' recorded was measured 
to be $3.5 \pm 1$K, which came from recording the total antenna temperature at the zenith to be $6.7^\circ$K. The contributions
from atmospheric absorption, ohmic losses and back-lobe response contributed to this result, but left an unaccounted-for 
temperature \cite{PW}. The remaining antenna temperature was subsequently interpreted to be black-body radiation from a
previous high-temperature state by Dicke, Peebles, Roll and Wilkinson \cite{Int}. In this paper, it was proposed that if this
excess temperature was black-body radiation from a high-temperature state, then the expansion of the universe would serve to 
adiabatically cool the radiation to the point, where now, it is mainly observed isotropically at mm wavelengths.



\subsection{COBE Observations}
Once the CMB was discovered, cosmologists started looking for anisotropies in the CMB. These anistoropies are incredibly small, to the point where the CMBR is uniform to one part 
in 100000. These fluctuations in the temperature are believed to originate from inhomogeneities in the distribution of matter at the recombination epoch\cite{Wright}.
The Cosmic Background Explorer (COBE) 
satellite was launched in 1989 with 3 main instruments; the Differential Microwave Radiometer (DMR), Far-InfraRed Absolute 
Specrophotometer (FIRAS), and the Diffuse InfraRed Background Experiment (DIRBE) \cite{Leverington}. The DMR is designed to 
map anisotropies in the CMB on the  large angular scale. In it's first year of data, the results showed temperature fluctuations 
at an angular resolution of $7^\circ$ and at frequencies of 31.5, 53, and 90 GHz\cite{Smoot}. The data from the COBE DMR 
observations was then later confirmed by balloon-borne experiments observations, measured at a higher frequency (170 GHz)\cite{Ganga}. 
\\The FIRAS instrument was designed to measure the black body spectrum of the CMBR. The findings presented by Mather \textit{et al} in 1994, 
gave an unprecented precision of the CMB temperature of $T = 2.726 \pm 0.010$K \cite{Mather}. The discovery of the black-body 
spectrum gave further evidence of the Hot Big Bang model at the time, as the model predicted it's existence. The DMR and FIRAS 
findings later earned Smoot and and Mather a joint nobel prize in 2006. 

\subsection{WMAP}
The Wilkinson Microwave Anisotropy Probe (WMAP) spacecraft was launched in 2001 by NASA in order to map the CMB temerature differences. It 
operated similarly to how COBE measured the CMB, by measuring the temperature difference between two directions. The WMAP produced five full sky maps of the CMB at 
five different frequencies. The CMB maps show the temperature fluctuations as colour differences, where the signal from our galaxy is subtracted for clarity (Refer 
to Bennett \textit{et al} \cite{WMAP 2003}for the maps). The data from the WMAP observations played a key role in solidifying the $\Lambda$CDM model as the standard model 
of cosmology, for which we have six cosmological parameters that describe the nature of our universe. The significance in the WMAP data at the time, lies in it's improvement upon 
the previous observational studies such as COBE and BOOMERanG, by providing much finer detail and sensitivity. These improvements allowed cosmologists to measure and analyse the 
CMB anistropies to reveal the size, age, matter content and geometry of the universe, aswell as allowing us to see the primordial structures that allowed to formations in the
universe we see today. 



\subsection{Planck}
In 2009 the European Space Agency launched the Planck satellite with the main aim of measuring the spatial anisotropies of the CMB temperature. The improved performance of the satellite
allowed cosmologists to extract virtually all of the information available in the anisotropies of the CMB temperature \cite{Planck Early}. This extensive, collaborive mission was also
able to measure the polarisation of CMB anisotropies, providing a window into the thermal history of the early universe. The Planck satellite used the statistical properties of the anisotropies 
to define key cosmological parameters with incredible accuracy \cite{Hist}. The 2018 results of the Planck collaboration showed that the $\Lambda$CDM model continued to provide an excellent fit 
to observational findings, with five of the six parameters were measured by Planck to better than 1\%\cite{Planck Final}. The mission exceeded expectations, with the best measurements
of the anistropy spectra of the CMB coming from the Planck satellite. 

\section{CMB to $\Lambda$CDM}
From the observational and theoertical studies of the CMB, cosmologists have been able to establish a standard model of cosmology
known as the '$\Lambda$CDM' model, that is bases itself on a flat universe, and is dominated by the cosomological constant(dark energy), $\Lambda$, and 
cold dark matter (CDM). 
$\Lambda$, the energy density of space, was first added to Einstein's field equations, in 1917, in order to force his universe model to be static, but is used today
to explain the accelerating expansion of space.
'CDM' stands for cold dark matter, a hypothetical type of dark matter, of which observations show that appoximately 85$\%$ of all matter is dark matter.
CDM is detailed in 1984 in a review article by Blumenthal \textit{et al}\cite{CDM}. 
The combination of these two, form the $\Lambda$CDM model as a six parameter model, as it consists of \cite{WMAP 2013} the physical baryon density, $\Omega_{\text{b}}h^2$; 
the physical cold dark matter density, $\Omega_{\text{c}}h^2$; the dark energy density, $\Omega_\Lambda$;
the scalar spectral index, $n_\text{s}$; the curvature perturbations, $\Delta^2_R$; and the reionisation optical 
depth, $\tau$. It's often referred to as the standard model of cosmology, \cite{Planck XIII} as many observational studies have supported the model's 
descriptions of the cosmos' features. One of the most notable descriptions, would be that of the structure of the CMB from the discovery of it's anistropies in 1992 \cite{Smoot}, 
from which the model was primarily supported.
Later, from 2003-2015, the \textit{WMAP} and \textit{Planck} observational studies have managed to determine the parameters to extremely high precision \cite{Planck XVI}, which
subsequently supported the model further. While the standard model has weathered decades of observational studies, and still remains the best fit to observation of our 
known universe, it still struggles to hold up at a 
small scales and still leaves room for observational studies to further explore the nature of our universe. Through the use of the CMB, cosmologists have been able to establish a 
full-fledged cosmological model to describe the large scale universe, from it's creation to it's current day dynamics, proving that the CMB has to be one of the most 
important cosmological tools we have at our disposal.
















\begin{thebibliography}{99}
    \bibitem{PW} Penzias A A and Wilson R W, 1965 \textit{Astrophys.J.} \textbf{142} 419-421
    \bibitem{Nobel} Wilson R, 1978, The Cosmic Microwave Background Radiation (Nobel Lecture) 474-477
    \bibitem{Hist} Durrer R, 2015 \textit{Classical and Quantum Gravity} \textbf{32} 124007; arXiv:1506.01907
    \bibitem{Int} Dicke R, Peebles P, Roll P and Wilkinson D, 1965 \textit{Astrophys.J.} \textbf{142} 414-419
    \bibitem{Smoot} Smoot G F, Bennett C, Kogut A, Wright E, Aymon J \textit{et al}. 1992 \textit{Astrophys.J.} \textbf{396} L1-L5
    \bibitem{Leverington} Leverington D, New Cosmic Horizons: Space Astronomy from the V2 to the Hubble Space Telescope, 2000 
    \textit{Cambridge Univeristy Press}
    \bibitem{Ganga} Ganga K, \textit{et al}. 1993 \textit{Astrophys.J.} \textbf{410} L57
    \bibitem{Mather} Mather J C, Cheng E, Cottingham D, Eplee R, Fixsen D \textit{et al}. 1994 \textit{Astrophys.J.} \textbf{420}
    439-444
    \bibitem{CDM} Blumenthal G R, Faber S M, Primack J R, Rees, M J. 1984 \textit{Nature} \textbf{311} 517-525
    \bibitem{Planck XIII} Planck Collaboration, Planck XIII,  2016 \textit{Astononmy \& Astrophysics} \textbf{594} A13 https://arxiv.org/abs/1502.01589
    \bibitem{WMAP 2003} Bennett C L, Halpern, M, Hinshaw G, \textit{et al}. 2003 \textit{Astrophysical J. Suppl} \textbf{148} 1; arXiv:astro-ph/0302207
    \bibitem{WMAP 2013} Bennett C L, Larson D, Weiland J L, \textit{et al}. 2013 \textit{Astrophysical J. Suppl} \textbf{208} 20; arXiv:1212.5225
    \bibitem{COBE 4} Bennett C L, \textit{et al}. 1994 \textit{Astrophysical Journal} \textbf{464} L1; arXiv:astro-ph/9601067
    \bibitem{Planck XVI} Planck Collaboration, Planck XVI, 2014 \textit{Astronomy \& Astrophysics} \textbf{571} A16; arXiv:1303.5076
    \bibitem{Fixsen} Fixsen D J, 2009 \textit{Astrophysical Journal} \textbf{707} 916-920; arXiv:0911.1955 
    \bibitem{Wright} Wright E L. 2003 \textit{Cambridge University Press} 291; arXiv:astro-ph/0305591
    \bibitem{Planck Early} Planck Collaboration. 2011 \textit{Astronomy \& Astrophysics} \textbf{536} A1
    \bibitem{Planck Final} Planck Collaboration. 2018 \textit{Astronomy \& Astrophysis} \textbf{641} A1

    

\end{thebibliography}

\end{document}
